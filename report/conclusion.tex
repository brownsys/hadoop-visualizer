\section{Conclusions \& Future Work}
In this work we aimed to improve the understanding of the network interaction of 
complex DISC frameworks. Numerous public and private institutions are becoming 
significantly dependent on DISC frameworks to help analyze datasets that are 
growing at alarming rates. The amount of data collected is becoming difficult 
to process with current technology and it is clear that improvements must be 
made if we as a society want to continue leveraging these large data repositories. 
Through the use of modified topologies, priorities, schedulers, and/or protocols we will  
continue to see innovation in ``big data'' analysis.

This work is only the first few steps towards improving our understanding of the interaction 
of DISC frameworks and their network. There still remains a lot of work in this area to 
fully understand the stages of Hadoop's network interaction and which stages or categories
can benefit from various current networks research. Lastly, we also still need to better 
understand representative workloads and the different types of network usage that occurs along-side
DISC frameworks. This work also only analyses Hadoop, we are interested in extending this work
to other DISC frameworks such as Spark.
